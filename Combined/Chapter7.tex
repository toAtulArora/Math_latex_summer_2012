\documentclass[12pt]{article}
% version = 1.00 of latexdemo.tex 2011 Feb 01

% Atul Singh Arora
% BS-MS 2016 | IISER Mohali
% Summers, 2012
% Math Project

%I did not add this
\usepackage{html}
%for making the real number R symbol to work
\usepackage{amsfonts}
%for getting a pipe symbol to work
\usepackage[T1]{fontenc}
%for increasing usable page area
\usepackage[margin=0.8in]{geometry}
%for automatically skipping a line after each paragraph
\usepackage[parfill]{parskip}
%for using text within formulae
\usepackage{amstext}
%for inserting images
\usepackage{graphicx}
%for proper enumeration
\usepackage{enumerate}
%for math theorem stuff!
\usepackage{amsmath, amsthm, amssymb}
\rmfamily
\begin{document}



\bibliographystyle{unsrt}  % define bibliography style



\begin{center}
\textsc{{\huge Symmetry\\}
More Group Theory\\
\small SP Status\\}
\begin{minipage}{0.4\textwidth}
\begin{flushleft} Atul Singh Arora \end{flushleft}
\end{minipage}
\begin{minipage}{0.4\textwidth}
\begin{flushright} {\small July 8-12, 2012} \end{flushright}
\end{minipage}
\\
\end{center}
\hrule
\vspace{12pt}
% \textbf{The concept of poles:}
This document contains record of my understanding of Chapter 7 More Group Theory, from Artin. I have addressed it to future me so the word choice may not suit everyone.
\par
Areas marked with a {\bf Doubt} or {\bf Find out} are ones I am not absolutely clear about. Perhaps reiterating later would help.\\
\hrule
\vspace{12pt}
\begin{flushright} {\small July 8, 2012} \end{flushright}
\par
\textsc{Corollary 5.1.28 } Let $M$ be the matrix in $SO_{3}$ that represents the rotation $\rho _{(u,\alpha)}$ with spin $(u,\alpha)$. Now let $B$ be another element of $SO_{3}$, and let $u' = Bu$. The conjugate $M'=BMB^{T}$ represents the rotation $\rho _{(u',\alpha)}$.\\
\par
\textbf {\textsc{7.4 The Class Equation of the Icosahedral Group}}
\par
Let $\theta=2\pi / 3$. Rotation by $\theta$ about a vertex $v$, represented by $\rho_{(v,\theta)}$, $\in I$, the icosahedral group (the group of rotational symmetries of a dodecahedron). If $v'$ is another vector, then rotation about this vector, represented by $\rho_{(v',\theta)}$ can be related to the rotation $\rho_{(v,\theta)}$ if in accordance with corollary 5.1.28, we could find a suitable $B$. But the vertices form different orbits under a given rotation. To transform $v$ to $v'$, both simply need to be in the same orbit for some rotation $\rho$ and this indeed happens, as can be seen geometrically (and can also be derived with mathematical rigour). However, it's easier to note that the 20 vertices of a dodecahedron, form a single orbit under the action of I. So always, $\exists$ a $B \in I$ s.t. $v'=Bv$.\\
The interest in this discussion arises form the conjugation relation between rotations. The existence of $B$ makes $\rho{(v,\theta)}$ and $\rho{(v',\theta)}$ conjugate elements. So if we take a rotation $\rho{(v,\theta)}$ and conjugate it with any element $B \in I$, then the result is also a rotation. Similarly if we take a vertex $v$ and operate it with all $B$, it generates the orbit of order 20. Since the only spins that can represent the same rotation as $(v,\theta)$ must be $(-v,-\theta)$ and since $-\theta \neq \theta$, therefore the number of elements in the conjugacy class of $\rho{(v,\theta)}$ is same as the number of elements in the orbit of $v$ under action of $I$, viz. $20$.
\par
Using the same analysis, we can take $\theta = 2\pi / 5$ for faces and conclude with the same reasoning, the number of elements in its conjugacy class to be $12$. When $\theta = \pi$ for centre of edges, we can use the same reasoning, but taking caution to account for the fact that rotation by $\pi$ is the same as rotation by $-\pi (=-\pi + 2 \pi =\pi)$. So the number of distinct rotation elements will be half the number of edges, viz. $15$.
\par
{\bf Find out: } Why do we have $\theta = 4\pi / 5$ included without which the count goes wrong. And why don't we have other angles, since $4\pi/5$ is a multiple of $2\pi/5$.
\par
The class equation of the icosahedral group then becomes\\
\begin{equation}
60 = 1 + 20 + 12 + 12 + 15
\label{IcosahedralClass}
\end{equation}
\par
\textsc {Simple Groups}
\par
Groups that do NOT contain proper normal subgroups, i.e. no normal subgroup other than $<1>$ and $G$.\\
Cyclic groups of prime order don't contain any proper subgroup and are hence simple.
\par
\textsc {Lemma 7.4.2 } Let $N$ be a normal subgroup of a group $G$.
\begin{enumerate}
\item If $x \in N$ then, $C(x) \in N$ (by definition of normal subgroup and conjugacy class)
\item $N$ is a union of conjugacy classes (for each element, there would be a conjugacy class)
\item The order of $N$ is sum of order of the constituent conjugacy classes. (summing the number of elements)
\end{enumerate}
\par
\textsc {Theorem 7.4.3 } The icosahedral group I is a simple group.
\par
Let's assume there exists a proper normal subgroup of $I$. It's order must be a proper divisor of $60$. Further, as follows from the lemma, the order of the subgroup must equal the sum of some of the terms on the RHS of \ref{IcosahedralClass}, but necessarily including the term $1$ (order of conjugacy class of the identity element). A look at the elements reveals that the sum can't be made divisible. Thus the assumption was wrong, the group must be simple.\\
\par
\begin{flushright} {\small July 9, 2012} \end{flushright}
<I have skipped a big chunk for I didn't have computer access when I studied it>\\
\textsc {post Theorem 7.5.4 (statement) } $A_{2}$ is trivial, $A_{3}$ is cyclic with prime order, only elements being $({\bf 1\,2\,3})$ and $({\bf 1\,3\,2})$ apart from identity, so it contains no subgroup, let alone a normal subgroup. $A_{4}$ has a kernel for homomorphism from $S_{4} \rightarrow S_{3}$ which lies in the alternating group, and is hence a proper normal subgroup (see 2.5.13). This makes $A_{4}$ a non-simple group.
\par
\textsc {Lemma 7.5.5}
\begin{enumerate}
\item For $n \geq 3$, $A_{n}$ is generated by 3-cycles.
\item For $n \geq 5$, the three cycles form a single conjugacy class in $A_{n}$
\end{enumerate}
\begin{proof}
The first is supposed to be anologous to the method of row reduction. Given any even permutation $p$, not the identity, that fixes $m$ of the indices, we can left multiply a suitable 3-cycle $q$, so that the product $qp$ fixes atleast $m+1$ indices. Don't worry we haven't yet shown this. It can be easily understood by considering the following. 
\par
Let $p$ be a permutation, other than identity. Now it will either contain a k-cycle with $k \geq 3$ or a product of atleast two 2-cycles. Since numbering the indices doesn't change anything, we suppose $p=({\bf 1\,2\,3\,...\,k})...$ or $p=({\bf1\,2})({\bf3\,4})...$. Now let $q=({\bf 3\,2\,1})$. Calculate $qp$ and you'll see something startling. The product fixes the index {\bf 1}. Why that works can be observed from the simple fact that whatever {\bf1} is mapped to in $p$, gets mapped back to {\bf 1} in $q$. That simple!
\par
Now for the second, more interesting part. Suppose $n \geq 5$. Now we've to show that for a given $q$ say $({\bf1\,2\,3})$, the conjugacy class is contained in $A_{n}$. What we already know is that $C(q) \in S_{n}$. So for some other 3-cycle, $q'$, $\exists\,\,p$, s.t. $q'=pqp^{-1}$. Now $p$ can either be even or be odd. If it's even, then $p \in A_{n}$. However if $p$ is odd, then we need to come up with some element $p' \in A_{n}$ (basically p' is even) such that $p'qp'^{-1} = q'$.\\ Let $\tau =({\bf 4\,5})$ which $\in S_{n}$ since $n\geq 5$. It's clear that $\tau q \tau^{-1} = q$. Replace $q$ in the conjugation with the aforesaid equation and you'll get $q'=pqp^{-1}=p\tau q \tau ^{-1} p^{-1} = (p\tau) q (p\tau)^{-1}$. Since (and quite cleverly so), $p\tau$ is now even, we have shown that the entire conjugacy class $\in A_{n}$.
\end{proof}
\par

% \theoremstyle{definition}
% \newtheorem{sect}{Section}
% \newtheorem{mydef}{Definition} 
% \newtheorem{thm}{Theorem}[sect]
% \begin{sect}{The Class Equation of the Icosahedral Group}
% \end{sect}

% \begin{thm}{Interesting Lemma}
% For every $n \geq 5$, $A_{n}$ is a simple group.
% \end{thm}

\textsc {Theorem 7.5.4 } For every $n \geq 5$, $A_{n}$ is a simple group.
\par
\begin{proof}
The proof is the perfect balance between interesting and simple. Look it in the book and there's really nothing much to explain, but its strategy is fairly interesting. <TODO: Complete this section should time permit>
\end{proof}
\begin{flushright} {\small July 10, 2012} \end{flushright}
\textbf {\textsc {7.6 Normalizers }} The stabilizer of orbit of a subgroup $H$ of a group $G$ for the operation of conjugation by $G$ is called the normalizer of $H$, denoted by $N(H)$.
\par
$N(H) = {g\in G | gHg^{-1} = H}$
\par
\textsc {Proposition 7.6.3 } Let $H$ be a subgroup of $G$, and let $N$ be the normalizer of $H$. Then
\begin{enumerate}[(a)]
\item $H$ is a normal subgroup of $N$.
\begin{proof} $gHg^{-1}=H\,\, \forall \,\,g \in N(H)$. And this follows from the definition of $ N(H) = \{ g \in G \,\,|\,\, gHg^{-1}=H $ \}
\end{proof}
\item $H$ is a normal subgroup of $G$ if and noly if $N=G$
\begin{proof} For $H$ to be a normal subgroup, $gHg^{-1}=H\,\, \forall \,\,g \in G$. So obviously, for that $N(H)=G$
\end{proof}
\item $|H|$ divides $|N|$
\begin{proof}follows from (a)\end{proof}
$|N|$ divides $|G|$
\begin{proof}follows from the fact that $N(H)$ is a stabilizer of $H$, and the counting formula\end{proof}
\end{enumerate}

\textbf { \textsc {7.7 The Sylow Theorems}}
\par
{\small Notation used: $a \mid b$ means $a$ divides $b$. $a \nmid b$ means the negative of the statement.}
\par
\textsc {Sylow $p$-subgroups}
\par
Let $G$ be a group of order $n$, and let $p \mid n$ where $p$ is prime. Let $p^{e}$ be the largest power of $p$ that divides n, i.e.
\begin{equation*}
n=p^{e}m
\end{equation*}
where $m$ is an integer and $p \nmid m$. Subgroups $H$ of $G$ with order $p^{e}$ are called \emph {Sylow $p$-subgroups of G}. Invoking the counting formula for the Sylow $p$-subgroup shows that these subgroups are $p$-groups whose index in the group is not divisible by $p$.
\par
\textbf {\textsc {The Theorems}}
\par
Let $G$ be a finite group whose order is $n$. For a given prime $p$ if $p \mid n$, then
\par
\textsc {{\small Theorem 7.7.2} First Sylow Theorem } $G$ contains a Sylow $p$-subgroup.
\par
\textsc {{\small Theorem 7.7.4} Second Sylow Theorem } 
\begin{enumerate}[(a)]
\item The Sylow $p$-subgroups of $G$ are conjugate subgroups.
\item Every subgroup of $G$ that is a $p$-group is contained in a Sylow $p$-subgroup.
\end{enumerate}
\par
\textsc {{\small Theorem 7.7.6} Third Sylow Theorem } say $n=p^{e}m$, where $p \nmid m$ and let $s$ denote the number of Sylow $p$-subgroups. Then $s \mid m$ and $s \equiv 1 \mod{p}$, i.e. $s = kp + 1$, for some integer $k$.\\
\par
Before getting into their proofs, let us look at some corollaries.
\par
\textsc {Corollary 7.7.3 {\small of the First Sylow Theorem }}
$G$ contains an element of order $p$.
\begin{proof}
Let $H$ be a Sylow $p$-subgroup. Consider an element $x \neq 1 \in H$. Since $G$ is finite, the subgroup $<x>$ (of $H$) will be finite. Also, the order of $x=|<x>|$. Invoking the counting formula, we know $|<x>|$ divides $|H|$. This means that order $x$ must also divide $|H|$. So order of $x$ must be a positive power of $p$, say $p^{k}$.\\
Then $x^{p^{k}}=1$, which means $x^{p^{k-1}\times p}=1$ $\Rightarrow x^{p^{k-1}}$ has order $p$.
\end{proof}
\par
\textsc {Corollary 7.7.5 {\small of the Second Sylow Theorem }} $G$ has exactly one Sylow $p$-subgroup if and only if that subgroup is normal.
\begin{proof}
Using the Second Sylow Theorem, its clear that since the $p$-subgroups are conjugates, if the conjugates are equal, i.e. $p$-subgroup is normal, then all $p$-subgroups would be the same. Hence exactly one $p$-subgroup would exist.
\end{proof}
\par
Now we begin with two lemmas required for the proof of the first Sylow Theorem.
\par
\begin{flushright} {\small July 11, 2012} \end{flushright}
\textsc {Lemma 7.7.9} Let $U$ be a subset of a group $G$. The order of the stabilizer Stab($[U]$) of $[U]$ for the operation of left multiplication by $G$ on the set of its subsets divides both of the orders, $|U|$ and $|G|$.
\par
Supplementary Explanation of the statement: Carefully understand section 6.10 (Operations on Subsets), everything used here, including notation makes perfect sense. If it doesn't, read section 6.8 (The operation on Cosets). I got stuck here for a while until clarity was recovered or to be accurate attained.
\begin{proof}
If $H$ is a subgroup of $G$, the $H$-orbit of an element $u$ of $G$ for left multiplication by $H$ is the right coset $Hu$. Let $H$ be the stabilizer of $[U]$. [{\bf Doubt |} What is the bracket notation supposed to mean? Is this the set of subsets? {\bf Clarification} The bracket notation implies $U$ is considered an element of the set of subsets.] Then multiplication by $H$ permutes the elements of $U$, so $U$ is partitioned into $H$-orbits, which are right cosets (why that happens is because if an element say $u_{1} \in U$ can be changed into another, say $u_{2} \in U$ by left multiplication by some $h \in H$, both would belong to the same orbit. If for no $h \in H$ this happens, then, they, despite being in the same set, would lie in different orbits). Now since each coset has order $|H|$, thus each orbit has order $|H|$. Now since orbits partition the set, and since each orbit is of the same size, $|U|=|$orbit$|\times ($number of orbits$)=|H|\times($number of orbits$)$, we know $|H|$ divides $|U|$. And since $H$ is a subgroup, by the counting formula (or more specifically, by Lagrange's Theorem) $|H|$ divides $|G|$.
\end{proof}
\textsc {Lemma 7.7.10} Let $n$ be an integer of the form $p^{e}m$, where $e>0$ and $p$ doesn't divide $m$. The number $N$ of subsets of order $p^{e}$ in a set of order n is not divisible by $p$.
\begin{proof}
$N$ is basically $^{n}C_{p^{e}}$ which is \\
\begin{equation*}
\left( \begin{array}{c}
n\\
p^{e} \end{array} \right) = \frac{n \, (n-1) \, ... \, (n-k) \, ... \, (n-p^{e}+1)}{p^{e}(p^{e}-1)\, ... \, (p^{e}-k)\, ... \, 1 }
\end{equation*}
$N \not \equiv 0 \mod{p}$ simply because every time $p$ divides a term $(n-k)$ in the numerator of $N$, it divides the term $(p^{e}-k)$ of the denominator just as many times (proved in just a moment).
\par
So for those who still don't understand why it makes any difference, consider $q$, $f_{1}$ and $f_{2}$ s.t. $p \nmid f_{1}$, $(n-k)=p^{q}f_{1}$ and $(p^{e}-k)=p^{q}f_{2}$ since the second term is divisible by $p$ as many times as the first. Now when you divide these, viz. first over the second, the result no longer has $p$ in the numerator and is hence not divisible by $p$ and since this happens for all possible numerator terms divisible by $p$, $p \nmid N$.
\par
Now for the proof of the statement, write $k$ as $p^{i}l$, where $p \nmid l$, then $i<e$. Replace this for $k$ and it makes both terms divisible by $p^{i}$ but not by $p^{i+1}$ [{\bf Find Out } Why the second assertion and how?]
\end{proof}
\begin{flushright} {\small July 12, 2012} \end{flushright}
We are now ready to prove the first Sylow theorem. Lets start.
\begin{proof}[Proof of the First Sylow Theorem]
What is given in Artin, is straight forward, yet for confirming I have understood, I am redoing the proof.
\par
\textsc {Strategy } We start by considering the set $\mathcal S$ of all subsets of $G$ of order $p^{e}$. If the theorem were assumed true, then one of these subsets will be the Sylow's $p$-subgroup. However, instead of finding this explicitly directly, we show that for some element of $\mathcal S$, say $[U]$, the stabilizer would have order $p^{e}$ and gotchya, that would be the Sylow's $p$-subgroup we intended to find.
\par
First thing we note here is that according to Lemma 7.7.9, we know that p will not divide the order of $\mathcal S$. Now we split the set $\mathcal S$ into orbits for the group action of left multiplication by $G$. Since orbits partition the set, we have
\begin{equation*}
N \text{ (as in the lemma) } = |S| = \sum\limits_{\text{orbits } O} |O|
\end{equation*}
Now obviously, at least one orbit must have an order which is not divisible by $p$. Let that orbit be $O_{[U]}$ of the element $[U] \in \mathcal S$. Let $H$ be the stabilizer of $[U]$. Now using the counting formula for orbit and stabilizer, we have $|G|=p^{e}m=|H|.|O_{[U]}|$. Since $|O_{[U]}|$ is not divisible by $p$, it must equal $m$ according to the equation. Thus, $|H|$ must be equal to $p^{e}$ and therefore $H$ my friend is the Sylow's $p$-subgroup.
\end{proof}

Before moving to the proof of the second Sylow's theorem, there's a `pseudo' lemma which is elementary, but must be proven for avoiding any confusion. It's as follows.
\par
\textsc {{\small Theorem 7.3.2} The Fixed Point Theorem } Let $G$ be a $p$-group and let $S$ be a finite set on which $G$ operates. If the order of $S$ is not divisible by $p$, there is a fixed point for the operation of $G$ on $S$ viz. $\exists$ a point $s$ whose stabilizer is the whole group.
\begin{proof}
Basically the proof requires the use of both counting formulae. The first says $|G|=$ |stabilizer| |orbit|. Now $|G|=p^{e}$ for some $p$ and $e$. The order of an orbit must be a number and therefore |stabilizer| would be a power of prime, and consequently, |orbit| would be $p^{k}$ for some $k\leq e$. So |orbit| is either 1 or a multiple of $p$.\\
Using the next formula, we break the set $S$ into orbits. We have
\begin{equation*}
\begin{array}{c}
|S|=|O_{1}|\,+\,|O_{2}|\,+\,|O_{3}|\,+\,...\\
\\
=\sum{\text{(orbits whose orders are multiples of }p)} + \sum{\text{(orbits with order 1)}}
\end{array}
\end{equation*}
Now since $|S|$ is not divisible by $p$, there must be at least one orbit with order 1. The stabilizer of this orbit will have order $p^{e}$ and must thus be the whole group. Therefore there must be at least one element in $S$ that is fixed under the action of $G$.
\end{proof}
Now we are good to go.
\begin{proof}[Proof of the Second Sylow Theorem] Basically same proof as Artin's, with different language and the `obvious' unsaid part explained
\par
\textsc {Strategy } For a given $p$-subgroup, say $K$ and Sylow $p$-subgroup, say $H$, both of $G$, we'll show that $K$ is contained in a conjugate $H'$ of $H$. That would prove part (b). For part (a), if $K$ is a Sylow $p$-subgroup, then $K$ equals $H'$ since $H'$ contains $K$ and both have the same order.
\par
We start with listing 3 desired properties of a subset of $G$, say $\mathcal C$. 
\begin{enumerate}[(a)]
\item $\mathcal |C|$ should not be divisible by $p$
\item Operation of $G$ on $\mathcal C$ should be transitive
\item $\mathcal C$ should contain an element, say $c$, whose stabilizer is $H$
\end{enumerate}
Now we must show that such a set exists. Well, the \emph{set} of left cosets of $H$, possesses all 3 properties. We better confirm that.
\begin{enumerate}[(a)]
\item The counting formula says $|G|=|H|$|number of cosets of $H$|\\
And by definition of $\mathcal C$, we have |number of cosets of $H$|=$| \mathcal C|$\\
Since by definition of Sylow $p$-subgroup, if we let $|G|=p^{e}m$, where $p \nmid m$, then $|H|=p^{e}$, therefore $| \mathcal C|$ is $m$ which means its not divisible by $p$.
\item Any coset of $H$ can be written as $gH$ for some $g \in G$. Thus for the action of $G$, all cosets of $H \in$ the same orbit. Thus, the action is transitive.
\item Every element of $h \in H$ is stabilized by $H$ because $H$ is a group. So the element $c \in \mathcal C$ which is fixed by $H$ is $[H]$.
\end{enumerate}
\par
Now the magic. Restrict the group action of $G$ to the $p$-subgroup $K$. Since $K$ is a $p$-subgroup, and $p$ doesn't divide $| \mathcal C|$ (property (a)), we can invoke the fixed point theorem to conclude that under the action of $K$, $\exists\,c' \in \mathcal C$ which remains fixed. \\
Also, the operation of $G$ is transitive on $\mathcal C$ (property (b)), therefore, $c'=gc$ for some $g \in G$. We also know that $H$ is the stabilizer of $c$ (property (c)), therefore $gHg^{-1}=H'$ (say) stabilizes $gc$ which is $c'$ itself (you can quickly verify this by seeing $gHg^{-1}gc=gHc=gc=c'$). Therefore, $H'$ contains $K$!
\end{proof}

\begin{proof}[Proof of the Third Sylow Theorem] This theorem has become very close to my heart, at least temporarily. Reason is a confusion which initiated because of my foolish assumption, viz. cosets are subgroups. Don't make that mistake and the proof would appear natural.
\par
As before, let $|G|=p^{e}m$ where $p \nmid m$. Let $H$ be a Sylow $p$-subgroup. From the counting formula, we can see that $m$ is the number of cosets of $H$, which is the same as the index $|G:H|$. So we have,
\begin{equation}
m=[G:H]=\frac{|G|}{|H|}
\label{third_1}
\end{equation}
Let $S$ denote the set of Sylow $p$-subgroups and let $s=|S|$, the number of Sylow $p$-subgroups. Now the stabilizer of a particular Sylow $p$-subgroup, say $[H]$ would be the normalizer $N(H)$, since according to the second Sylow theorem, the Sylow $p$-subgroups are conjugates. Also, $H$ is a normal subgroup of $N(H)$. This means $|H|$ divides $|N(H)|$, viz. 
\begin{equation}
|N(H)| \equiv 0 \mod{|H|}
\label{third_2}
\end{equation}
Now the number of Sylow $p$-subgroups, say $s$, would equal the number of elements in the conjugacy class of $H$. The stabilizer of $H$ under conjugation is $N(H)$ by definition. Using the orbit-stabilizer counting formula, we have,
\begin{equation}
s=[G:N(G)]=\frac{|G|}{|N(H)|}
\label{third_3}
\end{equation}
Using equations \ref{third_1}, \ref{third_2} and \ref{third_3}, we have
\begin{equation*}
m \equiv 0 \mod{s}
\end{equation*}
So this proves the first part, viz. $s$ divides $m$. The next part requires us to show that $s \equiv 1 \mod{p}$. We proceed by breaking $S$, the set of all Sylow $p$-subgroups, into $H$-orbits, for the action of conjugation by $H$. Now since $H$ is p-group, the order of any $H$-orbit should be a power of $p$. When the power is zero, the order will be 1, implying $H$ fixes the element. One such element is $[H]$. If we are able to show that it is the only element, then we'll have
\begin{equation*}
s=\sum (\text {multiples of } p) + 1
\end{equation*}
and therefore
\begin{equation*}
s \equiv 1 \mod{p}
\end{equation*}
Say $H$ stabilizes another element of $S$, namely $[H']$. Then $H \in N(H')$. Also, $H' \in N(H')$. Using the second theorem for Sylow-$p$-subgroups $H$ and $H'$ in $N(H')$ (whose order is greater than $p^{e}$ since it contains $H$, and is also divisible by $p$ as follows from the counting formula), $H$ must be expressible as a conjugate of $H'$, viz. $H=nH'n^{-1}$ for some $n \in N(H')$. However, $H'$ is normal in $N(H')$, hence $H=nn^{-1}H'=H'$. Thus $[H]$ is the only element stabilized by $H$. And that concludes the proof.
\end{proof}
\begin{flushright} {\small July 13, 2012} \end{flushright}
\textbf {\textsc {7.9 The Free Group}}
Here are some difinitions for reference. \emph {Free groups} are those whose generators satisfy no relation other than the ones implied from the group operation, for instance associativity. \emph {Free Semi-groups} are sets that are generated by generators with no relation as in free groups, but they don't have inverses. A word is called \emph {reduced} if no ruther cancellations can be made. If we start with $w$ and reduce it to $w_{0}$, then the latter is called the \emph {reduced form}.
\par
\textsc {Proposition 7.9.2} There's only one reduced from of a given word $w$.
\begin{proof}
We show that each method of cancellation is equivalent, and hence the given word $w$ has a unique $w_{0}$. Consier the length of $w$. If the length can't be reduced, then there's nothing to prove. If the length can be reduced, there exists some pair of symbols that can be cancelled, viz. 
\begin{equation*}
w=....xx^{-1}....
\end{equation*}
Let's consider the reduced form $w_{0}$. It obviously can't contain the pair $xx^{-1}$. Now there're two cases. First, the pair got cancelled at some stage during the process. So we can do that at any stage we please, without affecting the cancellation procedure. Second, one of them got cancelled at some stage as
\begin{equation*}
\begin{array}{ccc}
...\not x^{-1} \not xx^{-1}... & \text {or} & ...x^{-1} \not x \not x^{-1}...
\end{array}
\end{equation*}
Notice that the result is the same whether we cancel as above or cancel $xx^{-1}$, the pair. So this goes back to the first case. This shows that any two cancellation methods can be shown to be equivalent, and hence proves the proposition.
\end{proof}

$w \sim w'$ if they have the same reduced form, viz. $w_{0}=w'_{0}$.
\begin{flushright} {\small July 24 \& 25, 2012} \end{flushright}
\textsc {Proposition 7.9.3} Products of equivalent words are equivalent, viz. if $w~w'$ and $v~v'$ then $wv~w'v'$.
\begin{proof}
Essentially, we just note that we can convert both $w$ and $w'$ to $w_{0}$, its reduced form. Similarly for $v$ and $v'$ we have $v_{0}$. Now to prove $wv~w'v'$, we convert both to their reduced forms to obtain $w_{0}v_{0}~w_{0}v_{0}$ which proves the statement.
\end{proof}
\textsc {Definitions} for the proposition to follow:
\begin{enumerate}[(i)]
\item $S = {a,b,c..}$ is an arbitrary set of distinct symbols
\item $S' = {a,a^{-1},b,b^{-1},c,c^{-1},..}$ is the set consisting of symbols for every $a \in S$
\item $W'$ be a the semigroup of words made by juxtaposition of symbols from $S'$ and following the law of cancellation ($aa^{-1}=1$).
\end{enumerate}

\textsc {Proposition 7.9.4} The set $\mathcal F$ of equivalence classes of words in $W'$ is a group, with the law of composition induced from multiplication (juxtaposition) in $W'$.
\begin{proof}
Existence of identity and associativity of multiplication follows from $W'$. We just need to show the inverses exist. For any element $w \in \mathcal F$, the corresponding equivalent class (represented by the reduced form) would be expressable as the product $xy..z$ (arbitrary). The inverse of these would exist (by the definitions above) and thus the product $z^{-1}...y^{-1}x^{-1}$ would also exist. This is the inverse of $w$. Thus $\mathcal F$ is a group.
\end{proof}

\textbf {\textsc {7.10 Generators and Relations}}
\par
\textsc {Definition 7.10.1} A \emph {relation} among elements $x_{1},x_{2},x_{3}...x_{n}$ of a group $G$ is defined as a word $r$ in the free group of the set $x_{1},x_{2}....x_{n}$ that evaluates to $1$ in $G$.
\par
As an example, consider the dihedral group $D_{n}$, where $x$ represents rotation by angle $2\pi / n$ and $r$ represents a reflection about x-axis. Then from prior derivation, we know that (see page 164 of Artin's Text Second Edition) $x$ and $y$ will satisfy the following relations\\
\begin{equation*}
x^{n}=1,\,\,y^{2}=1,\,\,xyxy=1
\end{equation*}
Using these relations, one can convert any given element of $D_{n}$ into the form $x^{i}y^{j}$ where $0\leq i < n$ and $0\leq j<2$. (Note the third relation is the same as $yx=x^{-1}y$). This means we can derive the multiplication table for $D_{n}$ using these relations and thus these relations are also called \emph{defining relations}.
\par
Now it may not be simple to explicitly find the multiplication table, yet using the free group and a lemma, we will define a group generated by a given set of elements, with a given set of relations.
\par
\textsc {Lemma 7.10.3} Let $R$ be a subset of a group $G$. We can always find a unique smallest normal subgroup $N$ of $F$ that contains $R$. $N$ is called \emph {the normal subgroup generated by $R$}.
\begin{proof}
A non-empty subset of a group is a normal subgroup if and only if it is closed under law of composition (in our convention multiplication), inversion and conjugation (with an element of the given group) \& contains the identity element. [{\bf Confusion } What does conjugate of an element of $R'$ mean? {\bf Clarification} Its $grg^{-1}$ and not restricted to $r' r r'^{-1}$.] Thus we define $N$ to consist of those elements of $G$ which can be obtained from $R$ using a finite sequence of multiplication, inversion and conjugation. Uniqueness and being smallest follow readily.
\end{proof}
\begin{proof}[Alternate Proof]
We borrow the first statement from the previous proof. Let $R'$ consist of $r$ and $r^{-1}$ for every $r$ in $R$. An element of $G$ is in $N$ if it can be written as a product $y_{1}...y_{b}$ of arbitrary but finite length, where each $y_{v}$ is a conjugate of an element of $R'$. Showing $N$ is closed under multiplication is trivial. A little thought suffices to see inverses also exist. We show that the group is closed under conjugation also. Consider an element in $n$ in $N$. Now the group would be normal if $gng^{-1}$ is also in $N$. Since $n$ can be written as $y_{1}y_{2}...y_{r}$. Let it equal $g_{k_{1}}r_{1}g_{k_{1}}^{-1}\,g_{k_{2}}r_{2}g_{k_{2}}^{-1}\,...g_{k_{b}}r_{b}g_{k_{b}}^{-1}$. Now $gng^{-1}$ should also be in $N$. To prove this it would suffice to show that $gng^{-1}$ can also be expressed as a product $y'_{1}y'_{2}...$ for some arbitrary length (the symbol $y'_{v}$ also represents conjugate of an element or $R'$, but its used to avoid collision of variables). So, we have
\begin{equation*}
\begin{array}{lcl}
gng^{-1} & = & g(\,g_{k_{1}}r_{1}g_{k_{1}}^{-1}\,g_{k_{2}}r_{2}g_{k_{2}}^{-1}\,...g_{k_{b}}r_{b}g_{k_{b}}^{-1}\,)g^{-1} \\
 & = & (gg_{k_{1}}r_{1}g_{k_{1}}^{-1}g^{-1})\,(gg_{k_{2}}r_{2}g_{k_{2}}^{-1}g^{-1})\,...(gg_{k_{b}}r_{b}g_{k_{b}}^{-1}g^{-1}) \\
 & = & y'_{1}y'_{2}...y'_{b}
\end{array}
\end{equation*}
where $y'_{v}=gy_{v}g^{-1}=gg_{k_{v}}r_{v}g_{k_{v}}^{-1}g^{-1}=(gg_{k_{v}})\,r_{v}\,(gg_{k_{v}})^{-1}$. This proves that $N$ is closed under conjugation as well, and that in turn completes the proof.
\end{proof}
The empty set generates the trivial subgroup $\{1\}$.
\par
\textsc {Definition 7.10.4} Let $\mathcal F$ be the free group on the set $S=\{x_{1},...x_{n}\}$ and let $R=\{r_{1},...r_{k}\}$. Let $\mathcal R$ be the normal subgroup generated by $R$. Now the group generated by the set $S$ with the relations given by $R$ is the quotient group $\mathcal {G = F/R}$.
\par
The definition will seem natural upon realization of the fact that any element $r \in \mathcal R$ is mapped to the coset representing the identity element of $\mathcal G$. Also, a coset remains unchanged on multiplication with $r$. (Quick proof: Let $j\mathcal R$ be a coset. $rj\mathcal R=r\mathcal R j=\mathcal R j=j \mathcal R$.)
\par
The group $\mathcal G$ is also denoted by
\begin{equation*}
<x_{1},...,x_{n}|r_{1},...x_{k}>
\end{equation*}
\begin{flushright} {\small July 26 \& 27, 2012} \end{flushright}
\textsc {Example 7.10.7} {\small (This is given quite clearly in the text) } Let $T$ be the tetrahedral group (the group of symmetries of the tetrahedron), and let $x$ and $y$ denote rotations by $2\pi /3$ about the center of a face and about a vertex respectively. Let $z$ be rotation by $\pi$ about the center of an edge. Look at the text and the following relations will appear to hold naturally.
\begin{equation*}
x^{3}=1,\,y^{3}=1,\,z^{2}=1,\,xyz=1
\end{equation*}
Now following Artin, two questions arise.
\begin{enumerate}
\item Are these the defining relations for the group T? Which is the same as asking, is the group
\begin{equation*}
<x,y,z|x^{3}=1,\,y^{3}=1,\,z^{2}=1,\,xyz=1>
\end{equation*}
isomorphic to the group T?
\par
The answer's yes according to the text. We'll find out how in the next section.
\item {\small ({\bf Doubt} I don't understand the \emph {precise} meaning of the question)} How can one compute in a group ${\mathcal G} = <x_{1},...x_{n} | r_{1},...r_{k}>$ that's represented by generators and relations?
\par
Given the elements $x_{1},...x_{n}$, its simple to find the free group $\mathcal F$. Issue is to determine which element $w$ will represent identity in $\mathcal G$. As mentioned before, this will happen if and only if $w$ is in the normal subgroup $\mathcal R$. This apparently is known as the \emph {word problem} for $\mathcal G$. If the word problem can be solved, then we can decide when two elements of the free group represent equal elements of $\mathcal G$. To understand how, consider this; If $w_{1}=w_{2}$ in $\mathcal G$, then its the same as saying $w_{1}w_{2}^{-1}=1$ in $\mathcal G$. Now we know the only elements of $\mathcal F$ that represent identity in $\mathcal G$ are those that belong to $\mathcal R$, thus $w_{1}w_{2}^{-1}$ must belong to $\mathcal R$. Since we can solve the word problem, we already know which elements are in $\mathcal R$. We just need to separate them into the product of two elements, say $w_{a}w_{b}^{-1}$ and we'll know then that $w_{a}=w_{b}$ in $\mathcal G$.
\par 
When can or can't the word problem be solved, is according to the text, a question that requires more work to answer. It does assert however, that it can be solved in any finite group.
\end{enumerate}
\textsc {Example 7.10.11} This is very elementary to follow from the text, yet it is fairly clever and interesting. I am not redoing this here, since I understood it clearly in the first go.
\par
Before we can answer the first question, we must describe certain mapping properties. For that we first discuss the naming conventions that we'll use henceforth (this also is fairly elementary).
\par
There's a canonical homomorphism from $\mathcal F$ to $\mathcal G$, as with any quotient group, defined by $\pi$ as follows
\begin{equation*}
\pi : \mathcal {F \rightarrow F/R = G}
\end{equation*}
which sends $w$ of $\mathcal F$ to the coset $\overline w = [w\mathcal R]$. The kernel of the homomorphism is of course $\mathcal R$ (if its not obvious, revise section 2.12). According to the convention we've followed earlier, an overline like $\overline w$, represents the coset as an element, and thus the element belongs to $\mathcal G$, whereas $w$ by itself belongs to $\mathcal F$. However, this distinction is dissolved and we simply remember that $w_{1}$ and $w_{2}$ of $\mathcal F$ are equal in $\mathcal G$, if $w_{1}w_{2}^{-1}$ is in $\mathcal R$ (as mentioned earlier). So let's start.
\par
\textsc {Proposition 7.10.12} Mapping Property of the Free Group\\
Let $\mathcal F$ be the free group on a set $S=\{x_{1},x_{2},...\}$, and let $G$ be a group. Any map of sets $f:\,S \rightarrow G$ extends in a unique way to a group homomorphism $\varphi : \, \mathcal F \rightarrow G$. If we denote the image $f(x)$ of an element $x$ of $S$ by $\underline x$, then $\varphi$ sends a word in $S'=\{x_{1},x_{1}^{-1},x_{2},x_{2}^{-1},...\}$ to the corresponding product of the elements $\{\underline x_{1}, \underline x_{1}^{-1}, \underline x_{2}, \underline x_{2}^{-1},...\}$ in $G$.
\par
This is rather trivial but hinges (as pointed out in Artin) on the fact that the free group doesn't satisfy any other relation than those implied by the group axioms.
\par
\textsc {Proposition 7.10.13} Mapping Property of Quotient Groups\\
Let $\varphi : G' \rightarrow G$ be a group homomorphism with kernel $K$, and let $N$ be a normal subgroup of $G'$ that is contained in $K$. Let $\overline G' = G'/N$, and let $\pi : G' \rightarrow \overline G'$ be the canonical map $a \leadsto \overline a$. The rule $\overline \varphi (\overline a) = \varphi (a)$ defines a homomorphism $\overline \varphi : \overline G' \rightarrow G$, and $\overline \varphi . \pi = \varphi$.
\begin{proof}
First things first, the motivation for the definition of the map $\overline \varphi (\overline a) = \varphi (a)$. So let $a'=an$ for some $n\, \in \, N$, essentially just some element in the coset represented by $\overline a$. We define a new map, $\overline \varphi' (\overline a) = \varphi (a') = \varphi (an) = \varphi(a) \varphi (n) = \varphi (a) (= \overline \varphi (a) \text{ by definition})$. That justifies the definition.\\
Proving $\overline \varphi$ is a homomorphism, is now simple. 
\begin{equation*}
\begin{array}{lclr}
\overline \varphi (\overline a \overline b) & = &  \varphi(ab) & \text{ (definition) }\\
 & = &  \varphi(a) \varphi (b) &  (\varphi \text{ is a homomorphism) }\\
 & = &  \overline \varphi( \overline a) \overline \varphi ( \overline b) & \text{ (definition) }\\
\end{array}
\end{equation*}
Let's now prove that $\overline \varphi . \pi = \varphi$. Assume an arbitrary element $g' \, \in \, G'$. The map $\pi$ will send $g'$ to $\overline g'$ representing the coset $g'N$. Now the map $\overline \varphi$ will send $\overline g'$ to $\varphi (g')$, which is precisely the map $\varphi$.\\
\emph {Note:} The proof requires the group $N$ to be normal for else $\overline G'$ would not be a group and thus $\pi$ would cease to be a group homomorphism and everything falls apart.
\end{proof}
\textsc {Corollary 7.10.14}
\begin{enumerate}[(1)]
\item There's a canonical homomorphism $\psi: \mathcal G \rightarrow G$ that sends $x_{i} \leadsto x_{i}$ | \textbf {Doubt} (in ``..sends $x_{i} \leadsto x_{i}$''): So much so, that this seems incorrect to me!
\item $\psi$ is surjective if and only if the set $S$ generates $G$.
\item $\psi$ is injective if and only if every relation among the elements of $S$ is in $\mathcal R$.
\end{enumerate}
\begin{proof}
This section is driven by intuition yet Artin wasn't clear enough for me.\\
Let's start with the set $S'$ that generates $G$, in accordance with the prior discussion. Let $f$ be a map from $S'$ to $G$, viz. $f:\,S' \rightarrow G$. From Proposition 7.10.12, we know that $\exists$ a map $\varphi : \, \mathcal F \rightarrow G$. We already know that all relations $r_{i}$ in $R$ evaluate to identity in $G$. So $R$ is in the kernel of $\varphi$. Also, the kernel is a normal subgroup of $G$, and hence $\mathcal R$ must be contained in the kernel.\\
That motivates us to talk about the quotient group $\mathcal {G = F/R}$. From proposition 7.10.13, we know $\exists$ a homomorphism $\overline \varphi$ from $\mathcal G$ to $G$.\\
(This part has a little repetition, however that's necessary for the discussion as you'll soon discover) All relations $r_{i}$ in $\mathcal F$ (strictly speaking, by relation $r_{i}$ in $\mathcal F$ I mean the equivalence class corresponding to the reduced word of the relation, in $\mathcal F$), are mapped to the identity element $\mathcal G$ since its the quotient group $\mathcal {F/R}$. Further, since $\overline \varphi$ is a homomorphism, it maps identity to identity. Now the elements in $\mathcal F$ that are not in $\mathcal R$ but belong to the kernel of $\varphi$, will be mapped to their corresponding cosets in $\mathcal G$. So let $g_{\text{kernel}}$ be such an element in $\mathcal F$, then it is sent to its coset $g_{\text {kernel}}N$ in $\mathcal F$. Then it becomes quite natural to imagine what would happen if there weren't any such $g_{\text {kernel}}$, viz. kernel of $\overline \varphi = \{1\}$. Thus the map would be injective! (read Chapter 2 again if it doesn't sound obvious). So that proves part (3), since the only way kernel of $\overline \varphi = \{1\}$ is if all relations are included in $\mathcal R$.\\
Part (3) is obvious because if all symbols used in $G$ are included in $S$, then they will also exist in $\mathcal G$ (excluding those that are identity in $\mathcal G$, but they, obviously, wouldn't be present in $G$ either) and thus there would always be atleast one pre-image associated with every element of $G$, in $\mathcal G$.
\end{proof}

\vspace{500pt}
\hrule
\vspace{12pt}
\begin{center}
{\textsc {Remarks}}
\end{center}

\begin{flushright} {\small July 11, 2012} \end{flushright}
Lemma 7.7.9 and 7.7.10 despite being fairly elementary, did help me clearly understand the basics from the previous chapter, till the very last moment of documenting them.
\par
Why can't I see the obvious easily?\\
<TODO: Quote instances>
\begin{enumerate}
\item The main argument of Lemma 7.7.10
\end{enumerate}
\begin{flushright} {\small July 13, 2012} \end{flushright}
For some reason I just couldn't get myself to study today. Events from the past two days could have had this effect when I'd worked longer than I'd scheduled myself to.
\vspace{12pt}
\hrule

\end{document}

