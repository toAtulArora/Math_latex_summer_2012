\documentclass[12pt]{article}
% version = 1.00 of latexdemo.tex 2011 Feb 01

% Atul Singh Arora
% BS-MS 2016 | IISER Mohali
% Summers, 2012
% Math Project

%I did not add this
\usepackage{html}
%for making the real number R symbol to work
\usepackage{amsfonts}
%for getting a pipe symbol to work
\usepackage[T1]{fontenc}
%for increasing usable page area
\usepackage[margin=0.5in]{geometry}
%for automatically skipping a line after each paragraph
\usepackage[parfill]{parskip}
%for using text within formulae
\usepackage{amstext}

\begin{document}



\bibliographystyle{unsrt}  % define bibliography style



\begin{center}
\textsc{{\huge Symmetry\\}
Finite Subgroups of the Rotation Group\\
\small SP Status\\}
\begin{minipage}{0.4\textwidth}
\begin{flushleft} Atul Singh Arora \end{flushleft}
\end{minipage}
\begin{minipage}{0.4\textwidth}
\begin{flushright} {\small June 9, 2012} \end{flushright}
\end{minipage}
\\
\end{center}
\hrule

\vspace{12pt}


\textsc {Application of the Famous Formula: } The famous formula is:
\begin{equation}
\sum\limits_{i=1}^{k} (1 - \frac{1}{r_{i}}) = 2 \times (1 - \frac{1}{N})
\label{eqn.E}
\end{equation}
Recalling that $N$ is the order of the group which is not trivial, hence $N>1$. Also, $N$ is an whole number, and therefore the smallest value it can have is 2. So the RHS $\geq 1$. Also, as $N \to\infty$, the RHS $\to 2$, but remember N is finite. So effectively the $1\leq$ RHS $< 2$. Also, each term in the LHS $\geq \frac{1}{2}$, since $r_{i} \geq 1$.
\par
Now since the LHS must equal the RHS, there can't be more than 3 terms of LHS, else the sum would become $\geq$ 2, which the RHS can't reach for any value of $N$.
\par
Dividing this into 3 and classifying, we get\\
\emph{One orbit: }\\
So for a single orbit, $k=1$. So, the LHS becomes
\begin{equation}
1-\frac{1}{r} < 1
\end{equation}
while the RHS
\begin{equation}
2 \times (1 - \frac{1}{N}) \geq 1
\end{equation}
So this case is impossible.\\
\emph{Two orbits: }\\
For two orbits, we would have
\begin{equation}
(1 - \frac{1}{r_{1}}) + (1 - \frac{1}{r_{2}}) = 2 - \frac{2}{N}
\end{equation}
which is  the same as
\begin{equation}
\frac{1}{r_{1}} + \frac{1}{r_{2}} = \frac{2}{N}
\end{equation}
{\bf Doubt | } From this itself, Artin concludes that since $r_{i}$ divides $N$, the equation will hold only when $r_{1}=r_{2}=N$. I was unable to see why this was so. However, a little manipulation got me to the same result, but it still doesn't seem obvious to me. What am I missing?\\
Here's what I'd done.\\
Replaced $N$ once with $r_{1}n_{1}$ and one with $r_{2}n_{2}$ to get
\begin{equation}
\frac{1}{r_{1}} + \frac{1}{r_{2}} = \frac{1}{r_{1}n_{1}} + \frac{1}{r_{2}n_{2}}
\end{equation}
rearranged
\begin{equation}
\frac{1}{r_{1}}(1 - \frac{1}{n_{1}}) = \frac{1}{r_{2}}(\frac{1}{n_{2}} - 1)
\end{equation}
simplified
\begin{equation}
\frac{1}{r_{1}n_{1}}(n_{1} - 1) = \frac{1}{r_{2}n_{2}}(1- n_{2})
\end{equation}
since ${r_{1}n_{1} = r_{2}n_{2}}$
\begin{equation}
n_{1} + n_{2} = 2
\end{equation}
And since each orbit must contain atleast one element, $n_{i} \geq 1$. So the only possible solution is\\
$n_{1}=n_{2}=1$\\
$\Rightarrow r_{1}=r_{2}=1$.\\
So since there are only two poles, both fixed by all elements in $G$ hence, the only possibility (of the type of elements in the group) is rotation about a single axis, passing through both these poles (read points!).
\par
{\bf Doubt Context | }
Now as Artin says, is the most interesting case.\\
\emph{Three orbits: } What the text says till Case 1: $r_{1}=r_{2}=2$ and $r_{3}=k$ s.t. $N=2k$, is clear. For further clarity its given as\\
$r_{i}=2,2,k; \,\,\,\, n_{i}=k,k,2; \,\,\,\, N=2K$\\
It goes on to then say that there's one pair of poles ${p,p'}$ making the orbit $O_{3}$. So far so good as it readily follows from the value of $n_{3}$.\\ {\bf Doubt |} This is where I'm stuck.\\
It asserts, \emph{Half} of the elements of $G$ fix $p$, and the other \emph{half} interchange $p$ and $p'$.\\
Elephant in the room is, why Half?\\
This is what I had in mind, but I'm not sure.
\par
My Analysis:\\
Now we know that $O_{3}$, contains 2 elements since $n_{3}$ is 2. For a pole in this orbit, say $p$ as used above, $r_{p}=r_{3}$ [terms have the meaning as per their prior definition]. This means that the stabilizer of the pole, has order k and these are rotations about the axis passing through the origin and the pole (read point) $p$. Since there are only two poles, the other pole $p'$ must lie on this very axis. Thus, the same $K$ stabilizers, stabilize it. However the group has $2K$ elements. The other elements are NOT stabilizers and hence MUST interchange $p \text{ and } p'$. So they are 'reflections' which in $\mathbb R^{3}$ become rotations by $\pi$ about a line perpendicular to the line containing the poles. So half of them are fix $p$, other half interchange $p \text{ and } p'$.




%\being{equation}



\end{document}
