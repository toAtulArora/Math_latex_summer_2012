\documentclass[12pt]{article}
% version = 1.00 of latexdemo.tex 2011 Feb 01

% Atul Singh Arora
% BS-MS 2016 | IISER Mohali
% Summers, 2012
% Math Project

%I did not add this
\usepackage{html}
%for making the real number R symbol to work
\usepackage{amsfonts}
%for getting a pipe symbol to work
\usepackage[T1]{fontenc}
%for increasing usable page area
\usepackage[margin=1.0in]{geometry}
%for automatically skipping a line after each paragraph
\usepackage[parfill]{parskip}
%for using text within formulae
\usepackage{amstext}

\title{The Triangulation of Titling Data in
       Non-Linear Gaussian Fashion via $\rho$ Series}
\date{October 31, 475}
\author{John Doe\\ Magic Department, Richard Miles University
        \and Richard Row, \LaTeX\ Academy}
\begin{document}



\bibliographystyle{unsrt}  % define bibliography style



\begin{center}
\textsc{{\huge Symmetry\\}
Finite Subgroups of the Rotation Group\\
\small SP Status\\}
\begin{minipage}{0.4\textwidth}
\begin{flushleft} Atul Singh Arora \end{flushleft}
\end{minipage}
\begin{minipage}{0.4\textwidth}
\begin{flushright} {\small June 8, 2012} \end{flushright}
\end{minipage}
\\
\end{center}
\hrule

\vspace{12pt}

% \textbf{The concept of poles:}
Sir,\\
Writing Mathematics using the conventional windows tools like Word, Outlook etc. was becoming rather impossible and it had been resulting in a break of flow of my thoughts. Consequently I switched to LATEX and Sublime Text. Also I realized that when I start writing to you about a problem systematically, I usually end up solving it myself.\\
\par
The document henceforth would consist of the work I've done and understood. Things I haven't been able to understand, I would mark with a {\bf Doubt}, so that you can spot it easily and don't have to go through the entire text, while at the same time, providing you with the context so that you can reply accordingly.\\
\hrule
\vspace{12pt}

\textsc {Pole of Group Element: } Let $G$ be a finite subgroup of $SO_{3}$, of order $N > 1$. For now, consider only the rotation elements (of $\mathbb{R}^3) \in G$. Poles for such an element $h(\neq 1) \in SO_{3}$ are defined as the intersection points of the axis of rotation with the unit sphere $\mathbb{S}^2$. $h$ can't be identity since the definition of pole requires existence of an axis. Clearly, for each such $h$, there are 2 poles.
\par
\textsc {Pole of a Group: } Similarly in general, pole is defined for an element $g(\neq 1) \in G$. This pole can also be referred to as pole of the group.
\par
Thus, a pole of $G$, is a point, fixed by a group element $g \neq 1$.\\
\par
\textsc {G-Orbits and Poles: } According to Artin, and I quote ``The set $\mathcal P$ of poles of $G$, is a union of $G$-orbits. So $G$ operates on $\mathcal P$.''
\par
Here's the analysis of the second part of this statement. The set $\mathcal P$ of poles is basically a set of points. Each of these points can be ``operated'' upon by an element of $G$. In accordance with the definition of the word ``operate'' in terms of group action, we must verify that the point obtained after multiplication with an element $g(\neq) 1 \in G$ is also a pole. The other properties of group action are easy to verify.
\par
Let us prove it before continuing.\\ Consider a pole $p \in \mathcal P$.\\
Now $\exists\;h(\neq 1)$ s.t. $hp=p$ \hfill [from the definition of pole]\\
Let $g$ be an element of $G$. We have to show that $gp=q (say)$ is also a pole,\\
that is, $\exists\;k(\neq 1)$ s.t. $kq=q$\\
Lets replace $q$ with $gp$ in both sides, and $p$ with $hp$ in the RHS in the equation above.\\
So we need to solve for $k$ in the equation\\
$\Rightarrow kgp=ghp$\\
$\Rightarrow k=ghg^{-1}$\\
Since $G$ is a group, $ghg^{-1}$ exists and thus k exists, proving $gp$ to also be a pole of the group (specifically of $k$). Hence, we have shown $G$ operates on $\mathcal P$.
\par
Now for the first part of the statement. First, orbits are defined for a particular element of a set. However, if the set is the same as the orbit, then it needn't be specified.\\
{\bf Immediate Context before the doubt | }
In this case, let's fix a pole. The orbit of this pole under the group action would be the set of all poles obtained by operation of all $g \in G$. We just proved that each element of the group, when operates on a pole, creates another pole (of an element present in the group). Union of all the orbits must therefore be equal to $\mathcal P$, as $\mathcal P$ was defined to be the set of all poles of $G$.\\
{\bf Doubt | }
However, Artin has presented this in the other way as can be seen from the quote. What am I missing here?\\

\par
\textsc {Spin: } For a given $g \in G$, spin is the number of unordered pairs $(g,p)$, where $p$ is a pole of the group $G$.
\par
Since each rotation (excluding identity) has two poles, there are two unordered pairs associated, and thus the spin of each such rotation is two.\\
\par
\textsc {Deriving a Famous Formula: } The first objective here is to find a relation between the order of stabilizers for different $p$, and the order of the group, $G$.
\par
Now, for a given pole $p$, all elements $g \in G$, that leave $p$ unchanged, form a set $G_{H}$. $G_{H}$ is a cyclic subgroup and also a stabilizer by definition. The fact that its a subgroup can be verified easily as was done for the kernel. It is cyclic because the group $G$ is finite. It then follows that $G_{H}$ is generated by rotation by the smallest angle $\theta (>0)$ present in the group. If the order of the group $G_{H}$ is given by $r_{p}$, then 
$\theta = 2 \pi/r_{p}$.
\par
{\bf Doubt | } \emph{(Supplement explanation)} Is it correct to note here that the stabilizers $G_{H}$ for different poles $p$, will either form the same subgroup, or be disjoint (can be readily verified). The case would be former if and only if the poles are a result of intersection of the same axis. The latter would happen for all other cases. Also, if such subgroups were created for every pole $p$ of $G$, their union would exhaust the group $G$.\\
Now if all such stabilizers, minus their identity element, are made into a union, they would consequently contain twice as many elements as there are in (${G}$ minus the Identity element).  i.e.\\
\begin{equation}
\sum\limits_{p \in \mathcal{P}} {r_{p} - 1}= 2 \times ({|G|} - 1) 
\label{eqn.orderof_stab_group}
\end{equation}\\
It took me a while to get here. This relation is exactly the same as that given in Artin, but I would want to confirm if the reasoning is correct.
\par
\emph{(Textbook method explained)}
Now since $p$ is a pole, the stabilizer $G_{H}$ will contain at least one element other than identity, and thus $r_{p}>1$. Consequently $r_{p}-1$ elements (since we're excluding identity), stabilize $p$. Each of these elements thus has a spin two.\\
So every group element except the identity has two poles, implying its spin is 2. Taking $|G|=N$, there are $2(N-1)$ spins. Recalling that spin means the number of unordered pairs $(g,p)$ for a given $p$, total spins would be the total number of unordered pairs $(g,p)$. Now if we sum over all $r_{p} - 1$ (for excluding identity), then also we are counting all such pairs.\\
So the same relation (equation \ref{eqn.orderof_stab_group}) follows from this school of thought.\\
\par
So lets move forward. To simplify the LHS of equation \ref{eqn.orderof_stab_group}, we will use the counting formula.
\par
We already know from the counting formula (Size of a Group $G$= Order of coset of $H \times$ Number of Cosets) that
\begin{equation}
|G| \, = \, |G_{H}| \times | \text{Orbit of } G_{H}|
\label{eqn.B}
\end{equation}
since there is a bijective map between the orbit of $G_{H}$ and the cosets of $G_{H}$.
\par
Let $n_{p}$ denote $|$Orbit of $G_{H}|$. Rewriting both equations in terms of $N$, $r_{p}$ and $n_{p}$, we get\\
\begin{equation}
N=r_{p} \times n_{p}
\label{eqn.C}
\end{equation}
\par
Now we can see that if two poles $p$ and $p'$ are in the same orbit, then the order of their orbits is the same, i.e. $n_{p} =n_{p'}$. Equation \ref{eqn.C} demands the order of their Stabilizers to also be the same, i.e. $r_{p}=r_{p'}$.\\
Let's us arbitrarily denote different orbits by $O_{1}, O_{2}, ... O_{k}$. Now we note that if $n_{i}=n_{p}$ (note that the p is the same as it was in the previous context, a particular pole in the summing, although this result is not dependant on it) so by equation \ref{eqn.C} we have $r_{i}=r_{p}$.
\par
It is right here that I got stuck, which caused me to initiate writing like this in the first place. Now the trick here is to realize this very essential fact which is as follows.\\
We are summing over all poles $p$ in equation \ref{eqn.orderof_stab_group}. Now if $\exists\,p \text{   s.t. } p \in O_{i}$, then $\exists\,n_{i}$ poles in the orbit, each with the same number of stabilizers, i.e. $(r_{p} - 1) = (r_{i} -1)$ since $r_{p}=r_{i}$.\\ 
Now we can, from the left side of the equation, take out the contribution of all such poles to the total spin, and express it as $n_{i} \times (r_{i}-1)$. Also, each pole must belong to some orbit, therfore the entire sum (the LHS) may be written as
\begin{equation}
\sum\limits_{i=1}^{k} n_{i} \times (r_{i}-1) = 2 \times (N - 1)
\label{eqn.D}
\end{equation}
Take $r_{i}$ common from LHS and divide by N on both sides, to obtain
\begin{equation}
\sum\limits_{i=1}^{k} (1 - \frac{1}{r_{i}}) = 2 \times (1 - \frac{1}{N})
\label{eqn.E}
\end{equation}
\par
And that was the `famous' formula I'd never even seen before! As the book says, this formula might look small, but its a very strong tool.\\
\hrule
\par
The power of this function will be explored tomorrow.

%So basically what we're doing is, choosing the stabilizer as the subgroup, and orbit as the number of subgroups and multiplying them to find the order of the whole group.\\

%For the rotations of $\mathbb{R}^3$, poles are defined as the intersections of the axis of rotation with the unit sphere $\mathbb{S}^2$. Thus each rotation of $\mathbb{R}^3$ has two poles, except identity (as its axis is not unique).\\

%\vspace{12pt}
%So where does that leave us? We know the order of the stabilizer $G_{H} = r_{p} $. That's one down.\\
%From this we also know the number of elements that stabilize $p$, equals $r_{p} - 1$. Keep it mind, we'll use that in a bit.\\
%We have earlier shown that every element in the group $G$, corresponds to two poles, except the Identity Element. If we let $|G| = N$, then there are $2 \times (N-1)$ poles. So then the \\


\end{document}
