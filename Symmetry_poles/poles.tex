\documentclass[12pt]{article}
% version = 1.00 of latexdemo.tex 2011 Feb 01

% Atul Singh Arora
% BS-MS 2016 | IISER Mohali
% Summers, 2012
% Math Project

%I did not add this
\usepackage{html}
%for making the real number R symbol to work
\usepackage{amsfonts}
%for getting a pipe symbol to work
\usepackage[T1]{fontenc}
%for increasing usable page area
\usepackage[margin=0.5in]{geometry}
%for automatically skipping a line after each paragraph
\usepackage[parfill]{parskip}

\title{The Triangulation of Titling Data in
       Non-Linear Gaussian Fashion via $\rho$ Series}
\date{October 31, 475}
\author{John Doe\\ Magic Department, Richard Miles University
        \and Richard Row, \LaTeX\ Academy}
\begin{document}



\bibliographystyle{unsrt}  % define bibliography style



\begin{center}
\textsc{{\huge Symmetry\\}
Finite Subgroups of the Rotation Group\\
\small SP Status\\}
\begin{minipage}{0.4\textwidth}
\begin{flushleft} Atul Singh Arora \end{flushleft}
\end{minipage}
\begin{minipage}{0.4\textwidth}
\begin{flushright} {\small June 8, 2012} \end{flushright}
\end{minipage}
\\
\end{center}
\hrule

\vspace{12pt}

% \textbf{The concept of poles:}
Sir,\\
Writing Mathematics using the conventional windows tools like Word, Outlook etc. was becoming rather impossible and it had been resulting in a break of flow of my thoughts. Consequently I switched to LATEX and Sublime Text. Also I realized that when I start writing to you about a problem systematically, I usually end up solving it myself.\\
\par
The document henceforth would consist of the work I've done and understood. Things I haven't been able to understand, I would mark with a {\bf Doubt}, so that you can spot it easily and don't have to go through the entire text, while at the same time, providing you with the context so that you can reply accordingly.\\
\hrule
\vspace{12pt}

\textsc {Pole of Group Element: } Let $G$ be a finite subgroup of $SO_{3}$, of order $N > 1$. For now, consider only the rotation elements (of $\mathbb{R}^3) \in G$. Poles for such an element $h(\neq 1) \in SO_{3}$ are defined as the intersection points of the axis of rotation with the unit sphere $\mathbb{S}^2$. $h$ can't be identity since the definition of pole requires existence of an axis. Clearly, for each such $h$, there are 2 poles.
\par
\textsc {Pole of a Group: } Similarly in general, pole is defined for an element $g(\neq 1) \in G$. This pole can also be referred to as pole of the group.
\par
Thus, a pole of $G$, is a point, fixed by a group element $g \neq 1$.
\par
\textsc {G-Orbits and Poles: } According to Artin, and I quote "The set $\mathcal P$ of poles of $G$, is a union of G-orbits. So $G$ operates on $\mathcal P$."\\
Now the analysis of the second part of this statement. The set $\mathcal P$ of poles is basically a set of points. Each of these points can be 'operated' upon by an element of $G$. In accordance with the definition of the word 'operate' in terms of group action, we must verify that the point obtained after multiplication with an element $g(\neq) 1 \in G$ is also a pole. The other properties of group action are easy to verify.\\
Let us prove it continuing. Consider a pole $p \in \mathcal P$. Now $\exists g $ 


%For the rotations of $\mathbb{R}^3$, poles are defined as the intersections of the axis of rotation with the unit sphere $\mathbb{S}^2$. Thus each rotation of $\mathbb{R}^3$ has two poles, except identity (as its axis is not unique).\\

\vspace{12pt}

\bibliography{latexdemo.bib} % show the bibliography

\end{document}
